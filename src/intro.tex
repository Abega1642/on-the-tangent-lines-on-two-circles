\section*{Introduction}
\textit{Aristarchus of Samos}, an ancient Greek mathematician and astronomer, is widely recognized for his pioneering ideas in heliocentrism and his precise mathematical methods for estimating celestial distances. Living in the 3rd century BCE, Aristarchus proposed that the Earth orbits the Sun—an idea far ahead of its time and later foundational to Copernican theory. Despite the dominance of geocentric models in antiquity, Aristarchus’ work demonstrated an early understanding of large-scale spatial relationships in the cosmos, blending observational techniques with rigorous mathematical reasoning.\\

Among his notable contributions, Aristarchus’ treatise \textit{On the Sizes and Distances of the Sun and Moon}\cite{Aristarchus1572} presents one of the earliest known attempts to estimate astronomical distances using geometric methods. The work, preserved through \textit{Pappus of Alexandria}’s commentaries and later translated into Latin by \textit{Federico Commandino} in 1572, employs angular observations and proportional reasoning to derive estimates for the relative distances and sizes of celestial bodies. Aristarchus’ approach is deeply rooted in Euclidean geometry, utilizing circles, triangles, and tangent properties to construct his arguments.\\

Commandino’s translation and commentary significantly influenced Renaissance mathematicians, illustrating how ancient geometric reasoning could be applied to solve practical problems. The text remains a testament to the enduring relevance of classical geometry in scientific exploration.\\

The inspiration for this article emerged from my engagement with Aristarchus’ treatise while studying the Sun-Earth-Moon system. As I examined the geometric constructions employed by Aristarchus, I recognized their potential applications beyond astronomy, particularly in solving mechanical and geometric problems. Specifically, his use of tangent lines to circles and their intersections led me to consider a related problem in classical geometry: determining the common external tangents to two circles that intersect between them.\\

This article extends the geometric principles found in Aristarchus’ work by formulating and solving the problem of common tangents to two circles. By deriving explicit conditions for the existence and positions of these tangents, I aim to contribute to the ongoing dialogue between historical mathematical insights and contemporary problem-solving. Through this study, I highlight the continued relevance of ancient mathematical texts in inspiring new research in geometry and applied mathematics.
